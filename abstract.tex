\thispagestyle{empty}

\begin{center}
  {\bf \Huge Abstract}
\end{center}

\vspace{4cm}


\emph{
	Il recente sviluppo nella capacità di calcolo dei computer sta permettendo il passaggio delle telecomunicazioni da hardware dedicato ad algoritmi software. Esistono vari tipi di SDR che forniscono l'hardware necessario ad un computer.\\ L'obiettivo del lavoro è stato quello di implementare una tecnica di comunicazione largamente utilizzata quale OFDM per osservare il comportamento di un SDR venduto come decoder per radio e televisione digitale a meno di 8 euro.\\ Per la parte di trasmissione è stato utilizzato un SDR da laboratorio (Ettus B210). L'implementazione di OFDM è stata fatta utilizzando l'ambiente Gnuradio su cui è stato successivamente progettato un modulo aggiuntivo per la crittografia rsa.\\ I test eseguiti hanno confermato la validità di questi SDR ma ne hanno evidenziato anche alcuni limiti legati principalmete al loro hardware limitato (RTL2832). La qualità della comunicazione instaurata ha permesso di confermare l'efficacia di questi dispositivi in applicazioni reali.}
