\chapter{Conclusioni}
\label{cha:conclusioni}
\begin{itemize}
 \item Uno degli obiettivi del lavoro svolto è quello di analizzare il comportamento di un SDR economico in un utilizzo reale. Come prevedibile è stato necessario sfruttarlo al massimo delle potenzialità per ottenere una ricezione affidabile. La limitazione posta sulla frequenza di campionamento è appena sufficiente per superare il limite teorico sotto il quale non è possibile operare (Grafico 6.3). Durante il progresso del lavoro svolto è stato utilizzato anche un Ettus b210 per la ricezione (con le stesse variabili impostate nel sistema) che ha dimostrato di avere una notevole maggiore sensibilità e qualità dei campionamenti rispetto all'SDR-RTL. Queste piattaforme economiche sono definitivamente in grado di operare anche in ambienti con tecniche di trasmissione avanzate garantendo una buona ricezione del segnale.
 \item
 Dai test effettuati, facendo ascoltare i campioni audio ricevuti, è emersa un'interessante caratteristica sulla percezione umana della qualità di un suono. Infatti per piccole quantità di campioni persi (massimo 2-3\%) la qualità viene ancora considerata buona mentre non appena si supera la soglia del 4\%  viene rapidamente alterata a tal punto che sopra al 10\% non è possibile la comprensione delle parole.
 Questo risultato è perfettamente in linea con altri esperimenti riguardanti il dataloss consentito per uno streaming. \cite{dataloss}
 \item L'implementazione di OFDM in Gnuradio ha evidenziato un limite per quanto riguarda il sample-rate ottenibile. Per valori superiori a 4Msps la potenza di calcolo del processore sul singolo core fa da collo di bottiglia. Gnuradio dichiara di avere un sistema automatico in grado di distribuire il lavoro su più core quando è possibile. Una possibile spiegazione alla limitazione riscontrata potrebbe essere semplicemente una decisione presa da questo algoritmo automatico.
 \item Lo sviluppo dei blocchi per la crittografia RSA e la loro simulazione nell'ambiente Gnuradio hanno permesso di capire quali sono le problematiche da affrontare al fine di rendere questo sistema efficace in un utilizzo reale. Anzitutto le chiavi devono essere composte da 1024 - 2048 bit per non permettere ad un eventuale attaccante l'utilizzo di un attacco a forza bruta. La crittografia va fatta sull'intero blocco da 96byte scrivendo un algoritmo che faccia uso di una struttura dati per la memorizzazione dei byte necessari. \\Sarebbe possibile a tale scopo approfondire e utilizzare la predisposizione presente in Gnuradio per spedire vettori di byte.
 Nei sistemi di comunicazione moderni la codifica delle informazioni viene spesso implementata in una forma ibrida che consiste in uno scambio iniziale criptato con un algoritmo a chiave pubblica (ad esempio RSA) per la condivisione di una chiave simmetrica utilizzata poi per codificare le informazioni successive.
 \item  Il costo ridotto degli SDR-RTL e la quantità di tutorial a disposizione per imparare a ricevere qualche tipo di segnale permetterà a mio avviso un fenomeno simile a quello avvenuto nel campo dell'elettronica dopo l'introduzione di Arduino. Esso infatti si distingueva dai microcontrollori disponibili fino ad allora proprio per le medesime caratteristiche. Gli SDR-RTL hanno tutte le potenzialità necessarie per permettere lo stesso passo avanti, avvicinando giovani menti ancora inesperte a questo mondo interessante.
\end{itemize}


