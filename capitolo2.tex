\chapter{OFDM in Gnu Radio}
\label{cha:789}
La prima parte del lavoro svolto è consistita nell'implementazione di OFDM nell'ambiente GnuRadio. Il funzionamento di OFDM necessita di vari meccanismi complessi come l'assegnazione di informazioni alle sottoportanti, le correzioni in frequenza, l'equalizzazione e la correzione d'errore come precedentemente spiegato nella parte teorica. Lo svolgimento di queste operazioni sono rese possibili dalla presenza sia di blocchi generici utili ad esempio per la correzione d'errore che di blocchi disponibili specificamente per l'implementazione di OFDM. Per una comunicazione standard OFDM in Gnuradio non è necessaria la scrittura di algoritmi eventualmente importabili sotto forma di blocchi personalizzati, il lavoro consiste nel collegamento e nella configurazione dei parametri al fine di farli comunicare nella maniera corretta.
La comunicazione è divisa in due parti, una per la trasmissione che ha il compito di generare campioni per il driver dell'usrp-sdr ed una per la ricezione che partendo dai campionamenti effettuati dall'rtl-sdr decodifica le informazioni



